\section{Resumen ejecutivo}

El proyecto de optimización de bases de datos para Home Credit se destaca por
su enfoque exhaustivo en la gestión eficiente de grandes volúmenes de datos en
el sector financiero. Al adoptar SQL Server como el Sistema de Gestión de Bases
de Datos principal, se garantiza robustez y compatibilidad con estándares
comerciales. Desde la construcción de modelos E-R hasta la implementación de
funciones avanzadas como migración, seguridad y una interfaz web intuitiva,
cada fase del proyecto se ha abordado con precisión y atención al detalle.

La estrategia de migración, respaldada por un migrador eficiente en Go, ha
demostrado su capacidad para transferir datos a una velocidad significativa,
asegurando la coherencia y la integridad de la información. La inclusión de
características como un servidor de auditoría, funciones de CRUD, triggers y
niveles de acceso diferenciados refuerzan la seguridad de la base de datos,
contribuyendo directamente a un análisis de riesgo crediticio más preciso y
equitativo. La aplicación web desarrollada con Vue.js proporciona una interfaz
visualmente atractiva para la presentación de datos, consolidando así un
proyecto que no solo resuelve desafíos técnicos, sino que también sienta las
bases para una gestión de bases de datos efectiva y adaptable a las demandas
del mundo empresarial.

En términos de rendimiento, las pruebas han validado la capacidad de la base de
datos optimizada para manejar solicitudes simultáneas, marcando el éxito del
proyecto en cumplir con los estándares y requisitos detallados. En conclusión,
esta iniciativa representa un paso importante hacia la aplicación práctica de
habilidades técnicas para abordar problemas empresariales del mundo real,
particularmente en el ámbito financiero donde la precisión y la eficiencia son
cruciales.

\newpage

\section{Introducción}

En un entorno empresarial cada vez más orientado a los datos, la eficiente
gestión de bases de datos se convierte en un pilar fundamental para el éxito de
las organizaciones. Este proyecto, concebido en el marco del segundo semestre
de 2023, no solo busca ser un ejercicio académico, sino una oportunidad
concreta para aplicar habilidades técnicas y solucionar problemas del mundo
real.

La elección de SQL Server como SGBD principal no solo se basa en su robustez,
sino en su amplio uso en el ámbito comercial, proporcionando un terreno fértil
para la aplicación de conocimientos y técnicas avanzadas. El proyecto se
sumerge en el sector financiero, utilizando datos proporcionados por Home
Credit, una empresa que actualmente emplea métodos estadísticos y de
aprendizaje automático para realizar predicciones en el ámbito del riesgo
crediticio.

La importancia del análisis de riesgo crediticio radica en la capacidad de
otorgar préstamos de manera justa y sostenible. El proyecto busca no solo
aplicar metodologías de bases de datos avanzadas sino también contribuir a un
proceso que asegure que los clientes capaces de reembolso no sean rechazados
injustamente, y que los préstamos se otorguen con condiciones que maximicen el
éxito de los clientes.

Las secciones siguientes detallarán la estrategia para abordar cada uno de los
objetivos específicos, proporcionando una visión integral del proyecto desde su
conceptualización hasta la implementación práctica. Este proyecto no solo es un
desafío técnico; es un paso hacia la aplicación efectiva de habilidades para
resolver problemas empresariales del mundo real.

\section{Objetivos}

\subsection{Objetivo general}

Desarrollar un sistema de bases de datos integral utilizando SQL Server como
plataforma principal, abarcando desde la conceptualización de modelos (E-R),
modelos relacionales, esquemas, migradores, servidores de auditoría, funciones
de inserción, triggers, consultas SARGABLES, índices no clúster, hasta la
implementación de usuarios con diferentes niveles de acceso. El proyecto
incluirá la creación de una interfaz de usuario eficiente, ya sea web, móvil o
de escritorio, con un dashboard integrado.

\subsection{Objetivos específicos}

\begin{itemize}
    \item Desarrollar un modelo E-R preciso que sirva como base para la
        construcción de un modelo relacional coherente, asegurando la integridad y
        estructura adecuada de la base de datos propuesta.

    \item Crear esquemas organizativos para dividir las tablas de manera
        eficiente, junto con un migrador que permita actualizaciones sin
        comprometer la integridad de los datos. Implementar un servidor de
        auditoría para mejorar la trazabilidad y seguridad de las
        transacciones.

    \item Diseñar funciones y triggers que garanticen la consistencia y validez
        de la información ingresada. Además, desarrollar consultas SARGABLES y
        construir índices no clúster para optimizar el rendimiento de las
        consultas. Establecer usuarios con diferentes niveles de permisos y
        crear una interfaz de usuario eficiente, como un sistema web, móvil o
        de escritorio, con un dashboard capaz de cargar en un máximo de un
        minuto y manejar un millón de registros por componente, respaldado por
        una documentación completa que abarque todo el proceso de desarrollo.
\end{itemize}
